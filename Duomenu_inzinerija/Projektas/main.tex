\documentclass[10pt]{IEEEtran}

\usepackage[utf8x]{inputenc}
\usepackage[L7x]{fontenc}
\usepackage[lithuanian]{babel}
\usepackage{listings}

\lstset{
	basicstyle=\footnotesize,
	language=Java,
	morekeywords={String,each,in,Iterator}
}

\author{Maksim Norkin\\ \texttt{maksim.norkin@ieee.org}}
\title{Projektavimo šablonai \\ \footnote{Design Patterns}}

\begin{document}

	\maketitle

	\section{Įžanga}

		Programų inžinerijoje, projektavimo šablonas yra bendras, daug kartų naudojamas sprendimas nuolat kylančiai problemai išspręsti. Projektavimo šablonas nėra galutinis programinės įrangos sprendimas. Tai yra šablonas arba aprašymas kaip galima išspręsti kažkokią problemą skirtingose situacijose. Šablonus galima pervadinti į sprendimų rinkinius, kurie yra išgryninti programuotojų bendruomenės ir sąlygoja geriausią sprendimą. Objektiškai orientuoti projektavimo šablonai dažniausiai nurodo susiejimus ir sąveikas tarp klasių arba objektų, nesileidžiant į galutinių objektų ar klasių aprašymo. Daugelis šablonų įgyvendina objektiškai orientuotą arba paveldimą būseną, todėl jie gali būti neįmanomi pritaikyti funkciniame programavime. % paskutinio sakinio pabaigą reikia papildyti

		

\end{document}