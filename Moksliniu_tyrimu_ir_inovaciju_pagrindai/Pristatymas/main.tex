\documentclass{beamer}
\usepackage[lithuanian]{babel}
\usepackage[utf8x]{inputenc}
\usepackage[L7x]{fontenc}
\usepackage{lmodern}
\usepackage{caption}
\usepackage{subfig}
\usepackage{graphicx}
\usepackage{listings}

\newcommand*\oldmacro{}
\let\oldmacro\insertshortinstitute % save previous definition
\renewcommand*\insertshortinstitute{
  \leftskip=.3cm % before the author could be a plus1fill ...
  %\insertframenumber\,/\,\inserttotalframenumber\hfill\oldmacro}
  \oldmacro\hfill\insertframenumber\,/\,\inserttotalframenumber}

\let\oldshorttitle\insertshorttitle
\renewcommand*\insertshorttitle{
    \leftskip=0.4cm
\oldshorttitle\hfill Vilnius
}

\usetheme{Dresden}
\usecolortheme{vgtuef}

\logo{\includegraphics[height=20px]{img/isk_logo.jpg}}

\title[MapReduce]{MapReduce: Simplified Data Processing on Large Clusters}
\author[M. Norkin]{Maksim Norkin, ISIfm-13}
\institute[VGTU Fundamentinių mokslų faklutetas]{
  Vilniaus Gedimino technikos universitetas\\
  Fundamentinių mokslų fakultetas\\
  Informacinių Sistemų katedra\\
  \texttt{maksim.norkin@ieee.org}
}

\begin{document}

  \begin{frame}
    \titlepage
  \end{frame}

  \begin{frame}{Problema}

    \begin{itemize}
      \item Bendras protokolas didelių duomenų apdorojimui tarp atskirų sistemos mazgų
    \end{itemize}
  \end{frame}

  \begin{frame}{Darbo tikslas}
    \begin{itemize}
      \item Automatinis skaičiavimų lygiagretumas ir vykdymas tarp didelio kiekio skaičiavimo mazgų
    \end{itemize}
  \end{frame}

  \begin{frame}{Uždaviniai}
    \begin{itemize}
      \item Aprašyti programavimo modelį
      \item Aprašyti MapReduce įgyvendinimą 
      \item Aptarti galimus patobulinimus
      \item Pateikti įvykdymo greitaveikos matavimus
      \item Parodyti įgyvendinimo pavyzdį Google sistemose
      \item Aptarti panašius ir būsimus darbus
    \end{itemize}
  \end{frame}

  \begin{frame}{Išvados}
    \begin{itemize}
      \item Sėkmingas modelio panaudojimas Google sistemose
      \item Lengvas modelio panaudojimas
      \item Platus problemų sąrašas yra lengvai aprašomas per MapRecude skaičiavimus
      \item MapReduce algoritmo įgyvendinimas, kuris lengvai paskirstomas tarp tūkstančių skaičiavimo taškų
      \item Griežtas programavimo modelis leidžia lengviau skirstyti skaičiavimus tarp mašinų
      \item Tinklo pralaidumas yra esminė problema
    \end{itemize}
  \end{frame}

  \begin{frame}[allowframebreaks=0.8]{Vertinimas}

    \begin{itemize}
      \item Turinys ir pavadinimas
        \begin{itemize}
          \item Straipsnio pavadinimas tiesiogiai atitinka straipsnio turinį
        \end{itemize}
      \item Aktualumas
        \begin{itemize}
          \item Analizuojama problema yra labai aktuali
        \end{itemize}
      \item Argumentavimas
        \begin{itemize}
          \item Autoriai pateikia grafinius skaičiavimų įrodymus, pateikiamas konkretus veiksmų planas, kiekvienas žingsnis yra detaliai aprašomas ir apžvelgiamas
        \end{itemize}
      \item Metodika
        \begin{itemize}
          \item Modeliavimas, sisteminė analizė
        \end{itemize}
      \item Nuoseklumas
        \begin{itemize}
          \item Pradedama nuo metodo aprašymo, tuomet keliamasi į bendrą algoritmo įgyveninimą, aptariami galimi nesėkmių atvejai. Pateikiami realių problemų sprendimų pavyzdžiai.
        \end{itemize}
      \item Problema, tikslas, uždaviniai, išvados
        \begin{itemize}
          \item Iškelti uždaviniai išspręsti, 
        \end{itemize}
      \item Bloomo taksonomija
        \begin{itemize}
          \item Pasiekiamas Veiksmų plano lygmuo
        \end{itemize}
      \item Stilius
        \begin{itemize}
          \item Darbo stilius yra nuoseklus ir suprantamas kiekvieno, kas tik yra susidūręs su IT pobūdžio straipsniais
        \end{itemize}
      \item Literatūra
        \begin{itemize}
          \item Literatūros sąrašas stiprus
        \end{itemize}
    \end{itemize}

  \end{frame}

\end{document}