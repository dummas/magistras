\documentclass{beamer}
\usepackage[lithuanian]{babel}
\usepackage[utf8x]{inputenc}
\usepackage[L7x]{fontenc}
\usepackage{lmodern}
\usepackage{caption}
\usepackage{subfig}
\usepackage{graphicx}
\usepackage{listings}

\newcommand*\oldmacro{}
\let\oldmacro\insertshortinstitute % save previous definition
\renewcommand*\insertshortinstitute{
  \leftskip=.3cm % before the author could be a plus1fill ...
  %\insertframenumber\,/\,\inserttotalframenumber\hfill\oldmacro}
  \oldmacro\hfill\insertframenumber\,/\,\inserttotalframenumber}

\let\oldshorttitle\insertshorttitle
\renewcommand*\insertshorttitle{
    \leftskip=0.4cm
\oldshorttitle\hfill Vilnius
}

\usetheme{Dresden}
\usecolortheme{vgtuef}

\logo{\includegraphics[height=20px]{img/isk_logo.jpg}}

\title[Hadoop]{Hadoop\\Didelių duomeų apdorojimas studentišku biudžetu}
\author[M. Norkin]{Maksim Norkin, ISIfm-13}
\institute[VGTU Fundamentinių mokslų faklutetas]{
  Vilniaus Gedimino technikos universitetas\\
  Fundamentinių mokslų fakultetas\\
  Informacinių Sistemų katedra\\
  \texttt{maksim.norkin@ieee.org}
}

\begin{document}

  \begin{frame}
    \titlepage
  \end{frame}

  \begin{frame}{Susipažinimas}
    \begin{itemize}
      \item Programuotojas@Ruptela
      \item Projektai:
      \begin{itemize}
        \item Žmogaus eisenos atpažinimas
        \item Parkinsono ligos atpažinimas
        \item Akcijų biržos analitikų spėjimų analizė;
      \end{itemize}
    \end{itemize}
  \end{frame}

  \begin{frame}{Istorija}
    \begin{itemize}
      \item Google parašė popierių apie MapReduse \cite{Dean:2008:MSD:1327452.1327492};
      \item Doug Cutting and Mike Cafarella 2005 išleido pirmą Hadoop versiją Java pagrindu;
    \end{itemize}
  \end{frame}

  \begin{frame}{Hadoop}
    \begin{center}
      \includegraphics[width=50px]{img/hadoop.png}
    \end{center}
    \begin{itemize}
      \item Kas yra Hadoop?
      \begin{itemize}
        \item Norima atlikti kažkokius skaičiavimus su daug, \textit{daug} duomenų;
        \item Labai maži skaičiavimo resursai (keli seni kompiuteriai, renkantis dulkes ofiso kampe);
      \end{itemize}
      \item Susideda:
      \begin{itemize}
        \item HDFS
        \item MapReduce
      \end{itemize}
    \end{itemize}
  \end{frame}

  \begin{frame}{HDFS}
    \begin{itemize}
      \item Reali bylų sistema, parašyta Java pagrindu;
      \item Hadoop Distribuded File System;
      \item Lanksti masto didinime (angl. \textit{scalability});
      \item Pateikti pavyzdį ant lentos;
    \end{itemize}
  \end{frame}

  \begin{frame}{MapReduce}
    \begin{itemize}
      \item Lengvas sprendimas, norint apdoroti didelius duomenys su daug procesorių;
      \item Pagrindiniai tikslai, realizacijos metu:
      \begin{itemize}
        \item Klaidų tolerancija;
        \item Lanksti masto didinime (angl. \textit{scalability});
        \item Automatinis lygiagretumas ir paskirstymas;
      \end{itemize}
    \end{itemize}
  \end{frame}

  \begin{frame}{Programavimo modelis}
    \begin{itemize}
      \item I/O -- key/value rinkinys
      \begin{itemize}
        \item $<key,~value>$;
        \item $<userId,~profile>$;
        \item $<timestamp,~log~entry>$;
      \end{itemize}
      \item Programavimas vyksta naudojant dvi primityvias funkcijas
      \begin{itemize}
        \item map (in\_key, in\_value) $\rightarrow$ list(out\_key, intermediate\_value)
        \item reduce (out\_key, list(intermediate\_value)) $\rightarrow$ list(out\_value)
      \end{itemize}
      \item Pateikti pavyzdį ant lentos;
    \end{itemize}
  \end{frame}

  \begin{frame}{Privalumai}
    \begin{itemize}
      \item Labai paprastas ir lengvas naudoti;
      \item Jokios priklausomybės nuo schemos ir duomenų tipų;
      \item Jokio skirtumo nuo duomenų talpinimo lygio;
    \end{itemize}
  \end{frame}

  \begin{frame}{Trūkumai}
    \begin{itemize}
      \item MapReduce nėra aukšto lygio programavimo modelis;
      \item Jokios schemos ir indeksų;
      \item Mažas efektyvumas I/O pagrindu;
    \end{itemize}
  \end{frame}

  \begin{frame}[allowframebreaks=0.8]{Ekosistema}
    \begin{columns}
      \begin{column}{0.3\textwidth}
        \begin{itemize}
          \item Pig
          \item DataFlow
        \end{itemize}
        \includegraphics[width=100px]{img/yahoopig.jpeg}
      \end{column}
      \begin{column}{0.3\textwidth}
        \begin{itemize}
          \item Hive
          \item SQL
        \end{itemize}
        \includegraphics[width=100px]{img/hive.png}
        \vspace{45px}
      \end{column}
      \begin{column}{0.3\textwidth}
        \begin{itemize}
          \item Sqoop
          \item RDBMS
        \end{itemize}
        \includegraphics[width=100px]{img/sqoop.png}
        \vspace{110px}
      \end{column}
    \end{columns}

    \begin{itemize}
      \item Mahout
      \item Klasterizavimas, Savybių skaičiavimas
    \end{itemize}
    \begin{center}
      \includegraphics[width=100px]{img/mahout_logo.png}
    \end{center}

    \framebreak

    \begin{itemize}
      \item Zookeeper
      \item Sinchronizacija, monitoringas, grupinis servisas
    \end{itemize}
    \begin{center}
      \includegraphics[height=100px]{img/zookeeper.png}
    \end{center}

  \end{frame}

  \begin{frame}{Kas naudoja?}
    \begin{itemize}
      \item Adobe;
      \item Apple;
      \item Linkedin;
      \item Yahoo;
      \item Twitter;
      \item ...
      \item Beveik kiekvienas startup'as;
    \end{itemize}
  \end{frame}

  \begin{frame}{Ačiū}
    \begin{itemize}
      \item Klausimai?
    \end{itemize}
  \end{frame}

  \begin{frame}{Literatūra}
    \bibliographystyle{plain}
    \bibliography{references}
  \end{frame}

\end{document}