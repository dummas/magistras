\documentclass[10pt]{IEEEtran}

\usepackage[utf8x]{inputenc}
\usepackage[L7x]{fontenc}
\usepackage[lithuanian]{babel}
\usepackage{listings}
\usepackage{graphicx}
\usepackage{epstopdf}


\lstset{
    basicstyle=\footnotesize,
    language=Java,
    morekeywords={String,each,in,Iterator}
}

\author{Maksim Norkin\\ \texttt{maksim.norkin@ieee.org}}
\title{Modernios informacinės sistemos \\Dalis 2}

\begin{document}

    \maketitle

    \section{Laboratorinis darbas nr. 1}

        \section{Užduotis}

            Aprašyti IS projekto tikslus, problemas ir tų problemų sprendimus. 

            Aprašyti IS proejtko apimtį -- nurodyti kokie procesai bus kompiuterizuoti (pilnai arba dalinai), kaip tai bus daroma.

            Aprašyti numatomus organizacijos struktūros pateikimus įdiegus IS.

            Nurodyti kokie resursai bus reikalingi procesams atlikti (sudaryti resursų diagramą), aprašyti kaip skirsis resursų panaudojimas prieį ir po IS įdiegimo.

            Aprašyti, kokios technologijos bus naudojamos kuriant IS.

            IS projekto rizikos ir problemų nagrinėjimas -- įvertinti ir aprašyti numatomas IS kūrimo rizikas ir problemas.

        \section{Įvadas}

        \section{Analizė}

        \section{Išvados}

    \section{Laboratorinis darbas nr. 2}

        \section{Užduotis}

            IS užduotys (Use Case). Kiekviena užduotis turi turėti tisklą, prieš-sąlygas, post-sąlygas, alternatyvius srautus (jeigu tokie yra).

            Ne mažiau 3 užduočių.

        \section{Įvadas}

        \section{Analizė}

        \section{Išvados}

    \section{Laboratorinis darbas nr. 3}

        \section{Užduotis}

            IS architektūra -- nurodyti  IS sudėtines dalis, aprašyti kokias funkcijas jos atlieka.

        \section{Įvadas}

        \section{Analizė}

        \section{Išvados}

    \section{Laboratorinis darbas nr. 4}

        \section{Užduotis}

            Projekto valdymas. Darbų sąrašas, sąryšis tarp skirtingų užduočių -- Ganto diagrama, Ekonominė analizė (į šią analizę turi įeiti visų IS proejkto galimų išlaidų grupės).

        \section{Įvadas}

        \section{Analizė}

        \section{Išvados}

\end{document}